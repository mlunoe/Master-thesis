%!TEX root = thesis.tex
\chapter{Features to be Personalised} % (fold)
\label{ch:analysis}
This section analyses the user preferences and identifies which features should be modelled as personalisation constraints. It also seeks to define the default setting that consists the general model for a good editorial mix of a digital newspaper.

The focus of automatically generating the editorial mix introduces temporal, spacial and relational circumstances about the composition. But before these can be defined it is necessary to look at the user needs of the application.
%
 %These will consist the content constraints of the system along with those determined by user preferences. Therefore the constraints that belongs to composition conditions must be presented/defined.

Some prerequisites must be stated in order to set the focus of the problem, but also for the potential user to relate to the product. The research done by \cite[p. 1]{FULLTEXT01.pdf} and \cite[p. 6-7]{kristin_fredrik.pdf} determines the preferable size of the digital newspaper to be $14.732 \times 20.828$cm $\sim$ size A5, which reflects the size of the iPad. Therefore this project will target the iPad as its primary device. The handheld device also introduces mobility, which is also of great preference to the potential users.
%
%\begin{itemize}
	%\item $14.732 \times 20.828$cm $\sim$ size A5, which reflects the size of the iPad% \cite[p. 1]{FULLTEXT01.pdf} % \cite[p. 6-7]{kristin_fredrik.pdf}
	%\item reader knows about constraints
	%\item relevance feedback introduced (click on article, time spend reading it and scroll)
	%\item argument that transparency of the general model is enough for the user to provide settings \cite[p. 7]{gervasum2001ws.pdf}
%\end{itemize}

\section{User Needs}
This section will define the user needs for the application. A full description of personas, scenarios and business case is found in appendix~\vref{appendix:user-needs}.

In the explored literature users shows much interest in being able to turn pages as it is done in a regular newspaper. \cite[p. 6]{FULLTEXT01.pdf} describes this as ``open, turn pages, chose article, read and return''.

\cite{fulltext.pdf} proposes the use of collaborate filtering to handle the problem of converging, which is what will happen if no non-personalised articles are introduced, but this still only concerns articles that are within the area of the users interest. If e.g.\ a user has not shown interest in politics, the news of Barack Obama becoming the President of USA will never be included in the newspaper. Instead a ratio between personalised and general articles will solve this issue, and since it is not within everyones interest to receive general news, this ratio should be adjustable.

Users navigate the newspaper using sections and headlines as the main entry points \cite{FULLTEXT01.pdf} and these should therefore be kept in the digital version.

Users express that these should be put into menu \cite{kristin_fredrik.pdf}.

\cite{fulltext.pdf} proposes personalised excerpts from the articles to further ease the navigation.

\subsection{Non-functional Requirements}
\begin{itemize}
	\item open, turn pages, chose article, read and return
	\item both general and personal news
	\item full screen display of article
	\item graphical/textual content ratio
	\item opens in front page view (summery of newspaper few articles)
	\item most interesting articles on the front page
	\item put in personalised sections
	\item back page, funnies?
	\item navigation through section headlines, article headlines and article summaries / excerpts
	\item relevance value by article
	\item menu w. section headlines
	\item page numbers
	\item page turn
	\item possibility of relevance feedback
	\item keyword based user profile
	\item ability to adjust preference variables
	\item social network and comunity incorporation
\end{itemize}

\subsection{Functional Requirements}
Also some technical requirements are expressed by the users from the literature:
\begin{itemize}
	\item clear overview of content
	\item easy navigation
	\item contemplated typography and design
	\item columns should be divided to fit into one screen with possible images or videos
	\item familiarity in design and layout from printed paper
	\item news valuation, e.g.\ positioning of lead story
	\item continuous updates
	\item ability to search
	\item video and sound
	\item Hould be readable in both landscape and portrait
	\item touch screen interaction
	\item Functionality from online newspaper
	\item Name of columnist
	\item Transparency of implicit relevance feedback (see/modify current weights of categories)
	\item dynamic short-term + static long-term user profile
	\item relevance feedback
\end{itemize}

\section{Delimitation}
The \cite{DCMI} proposes 15 meta data elements, i.e.\ Title, Creator, Subject, Description, Publisher, Contributor, Date, Type, Format, Identifier, Source, Language, Relation, Coverage and Rights.

The initial approach involved computing the tf-idf similarity between documents and the user and the documents in between using the Python libraries for this \cite{NLTK}. This approach works on a bow (bag-of-words) with key words and weights representing a single item. The weight is computed by the number of occurrences in the provided text and a cosine distance determines the similarity. Python also provides an interface for working with WordNet -- a large lexical database of English words and their relationships in the form of different graphs. This opens the door to a more in-depth analysis of the optained news items. \cite{116262780379.pdf} presents an algorithm for enriching articles using WordNet's hypernym-graphs. A hypernym graph is generated by the top $20\%$ frequent keywords of an article and weighted by:
$$W(d, f) = 2 \cdot \frac{1}{1+e^{-0.125(d^3\frac{f}{TW})}}-0.5$$

Where $d$ stands for the node's depth in the graph (starting from root and moving downwards), $f$ is the frequency of appearance of the node to the multiple graph paths and $TW$ is the total number of words used to generate the hypernym graph.

In order to be able to work with hypernyms, the words must be converted to synsets. For each word there exists a synset for each use of the word, with the most frequently used first. Every synset is included at this point, but in a later stage this could be further focused by only using the top $n$. An analysis on how many percent of the words

\todonote{Which design choices to focus on?}
Introduce columns and remove pages to introduce newspaper like layout

Do not focus on summaries -- out of scope, 

Support images only, not video -- can easily be introduced.

Not implement support for social networking, but included in the proposed design

Predefined categories is not a full list, but are of the most recurring in popular news sites.

\todonote{In which period of time is an article relevant to a user? Maybe if it is still available, then it is still interesting -- new approaches or discussion about the subject might arise. How do we control that a news item is not missed? Keep index of what has been viewed in addition to what has been read.}


% section analysis (end)
