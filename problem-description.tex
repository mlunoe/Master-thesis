\documentclass{acm_proc_article-sp}

\usepackage[utf8]{inputenc} 
\usepackage[T1,OT1]{fontenc}
\usepackage{ae,aecompl}

\usepackage{url}
\usepackage{subfigure}
\usepackage{varioref}
\usepackage{mdwlist}
\usepackage{float}
\usepackage{multirow}

\usepackage{txfonts}

%\usepackage{times}
%\usepackage{atbeginend}
\usepackage[scriptsize]{caption}

\newcommand\writtenby[1]{
	\begin{flushright}
		\footnotesize{Written by: #1}
	\end{flushright}
}

\setlength{\belowcaptionskip}{0pt}

\def\sharedaffiliation{%
\end{tabular}
\begin{tabular}{c}}
\sloppy
\begin{document}
\title{Optimising the Editorial Mix for a Digital Newspaper using Constraint Programming}
\subtitle{}

\numberofauthors{1}
\author{
	\alignauthor Michael Lunøe\\
       \email{s062596@student.dtu.dk}
%
	\sharedaffiliation
       \affaddr{Technical University of Denmark, DTU Informatics}\\	
       \affaddr{Richard Petersens Plads, Building 321}\\
       \affaddr{DK-2800 Kgs. Lyngby, Denmark}
}
\maketitle

\section{Problem Description}
The overall idea is to combine Constrained Optimisation Problems (COPs) of Constraint Programming (CP)\footnote{In constraint satisfaction, constrained optimization seeks for a solution maximizing or minimizing a cost function, Wikipedia.} and personalisation\footnote{Personalization involves using technology to accommodate the differences between individuals, Wikipedia.}.
The project will be divided into two main areas; i.e. an assessment of the use of COPs in the context of personalisation and a direct application of this in the form of a personal digital news paper, where CP is used to personalise the content and layout of a digital newspaper.

\subsection{Personalisation Challenges}
In an attempt to personalise the content of a newspaper, the report will try to analyse which preferences the users will have with respect to the content of relevant articles for each page. It will describe the search for articles to fit within these pages as an COP and try to solve it. In the digital newspaper, articles of more importance should take up more space than articles of less importance and the problem of arranging each page will go through an equal solving process.

\subsection{Algorithmic Challenges}
To be able to use and assess CP and COPs in the context of personalisation a full understanding must be acquired. Advantages and disadvantages may each be listed and discussed along with a time and space analysis. Further- more additional meta heuristics\footnote{In computer science, metaheuristic designates a computational method that optimizes a problem by iteratively trying to improve a candidate solution with regard to a given measure of quality, Wikipedia.} may be considered to improve the solution. The findings of this analyses will be used to try to solve the personalisation problem described above.

\subsection{Existing solutions}
There already exists solutions to the problem of personalisation, but the role of CP and COPs will be analysed and hopefully new areas will be explored.

%Optimising the Editorial Mix for a Digital Newspaper using Constraint Programming
%Optimering af den Redaktionelle Sammensætning i en Digital Avis med brug af Constraint Programmering
%start 6/2
%aflevering 3/8
\newpage
\section{Objectives}
\subsection{General project objectives}
The overall objective of the project is to make the students familiar with Constraint Programming, especially Constraint Optimised Problems, and its uses in the con- text of personalisation. A specific personalisation problem is implemented and solved using Constraint Programming and relevant metaheuristics. Finally, the project pro- vides a conceptual basis for developing personal digital solutions and the uses for Constraint Programming.

\subsection{Learning objectives}
Explore existing literature and summarise it.

Analyse the applicability of personalisation in a given domain and derive features of the problem to be personalised.

Find applicable metaheuristics for personalisation problems and refine them accordingly.

Model the identified features as constraints in a Constrained Optimisation Problem. Solve the defined Constraint 

Optimisation Problem in Constraint Programming.

Evaluate Constraint Programming as a tool for personalisation problems and compare the solution to existing solutions.

Describe and discuss the project in a report and present it orally.

% Bibliography
%\bibliography{report}
%\bibliographystyle{plain}
%\clearpage
%APPENDICES are optional
%\balancecolumns
%\appendix
%\input{A_stories}

%\balancecolumns % GM June 2007
% That's all folks!
\end{document}