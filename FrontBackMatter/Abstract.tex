%!TEX root = ../thesis.tex
% Abstract
\pdfbookmark{Abstract}{Abstract}
\begingroup
\let\clearpage\relax
\let\cleardoublepage\relax
\let\cleardoublepage\relax

\chapter*{Abstract}
This paper proposes a Constraint Programming (CP) approach to personalise a composition of articles that follows rules of the editorial mix in a digital newspaper. Inspiration from conventional newspapers will be used to express the problem as a Constraint Optimisation Problem and solved using local search for CP taking advantage of both probabilistic and logical approaches. Furthermore, this paper proposes a keyword based solution, using WordNet enrichment of articles in combination with a comparison of entities, to determine the relevance of an article to a user defined topic and similarity between articles. As a by-product of the implementation a library for solving personalisation problems using CP was developed.

\vfill

%\selectlanguage{danish}
\pdfbookmark[1]{Resumé}{Resumé}
\chapter*{Resumé}
Denne afhandling præsenterer Constraint Programming (CP) anvendt til at personalisere sammensætningen af artikler i en digital avis, der følger redaktionelle regler. Med inspiration hentet fra konventionelle aviser bliver de redaktionelle regler udtrykt som et Constraint Optimisation Problem og løst vha. local search teknikker for CP. Dette gør at både probalistiske og logiske fordele bliver udnyttet. Udover dette bliver en keywordbaseret løsning præsenteret til at udregne relevans af de enkelte artikler ift. brugerdefinerede emner, men også artiklerne imellem. Løsningen gør brug af WordNet til at berige artiklerne og en sammenligning af entiteter til den semantiske analyse. Som et bi-produkt af implementeringen blev der udviklet et bibliotek til at løse personaliseringsproblemer vha. CP.

%\selectlanguage{british}

\endgroup			

\vfill

%\pdfbookmark[1]{Abstract}{Abstract} % Bookmark name visible in a PDF viewer
%
%\begingroup
%\let\clearpage\relax
%\let\cleardoublepage\relax
%\let\cleardoublepage\relax
%
%\chapter*{Abstract} % Abstract name
%
%Short summary of the contents\dots
%%\todo[inline]{Til Chris: Tusind tak for hjælpen! :)}
%
%\endgroup			

\vfill