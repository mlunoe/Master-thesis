%!TEX root = ../thesis.tex
% Table of Contents - List of Tables/Figures/Listings and Acronyms

\refstepcounter{dummy}

\pdfbookmark[1]{\contentsname}{tableofcontents} % Bookmark name visible in a PDF viewer

\setcounter{tocdepth}{1} % Depth of sections to include in the table of contents - currently up to subsections

\setcounter{secnumdepth}{3} % Depth of sections to number in the text itself - currently up to subsubsections

\manualmark
\markboth{\spacedlowsmallcaps{\contentsname}}{\spacedlowsmallcaps{\contentsname}}
\tableofcontents 
\automark[section]{chapter}
\renewcommand{\chaptermark}[1]{\markboth{\spacedlowsmallcaps{#1}}{\spacedlowsmallcaps{#1}}}
\renewcommand{\sectionmark}[1]{\markright{\thesection\enspace\spacedlowsmallcaps{#1}}}

\clearpage

\begingroup 
\let\clearpage\relax
\let\cleardoublepage\relax
\let\cleardoublepage\relax

%----------------------------------------------------------------------------------------
%	List of Figures
%----------------------------------------------------------------------------------------
%
%\refstepcounter{dummy}
%%\addcontentsline{toc}{chapter}{\listfigurename} % Uncomment if you would like the list of figures to appear in the table of contents
%\pdfbookmark[1]{\listfigurename}{lof} % Bookmark name visible in a PDF viewer
%
%\listoffigures
%
%\vspace*{8ex}
%\newpage
%
%%----------------------------------------------------------------------------------------
%%	List of Tables
%%----------------------------------------------------------------------------------------
%
%\refstepcounter{dummy}
%%\addcontentsline{toc}{chapter}{\listtablename} % Uncomment if you would like the list of tables to appear in the table of contents
%\pdfbookmark[1]{\listtablename}{lot} % Bookmark name visible in a PDF viewer
%
%\listoftables
%        
%\vspace*{8ex}
%\newpage
%    
%%----------------------------------------------------------------------------------------
%%	List of Listings
%%---------------------------------------------------------------------------------------- 
%
%\refstepcounter{dummy}
%%\addcontentsline{toc}{chapter}{\lstlistlistingname} % Uncomment if you would like the list of listings to appear in the table of contents
%\pdfbookmark[1]{\lstlistlistingname}{lol} % Bookmark name visible in a PDF viewer
%
%\lstlistoflistings 
%
%\vspace*{8ex}
%\newpage
       
%----------------------------------------------------------------------------------------
%	Acronyms
%----------------------------------------------------------------------------------------

\refstepcounter{dummy}
%\addcontentsline{toc}{chapter}{Acronyms} % Uncomment if you would like the acronyms to appear in the table of contents
\pdfbookmark[1]{Acronyms}{acronyms} % Bookmark name visible in a PDF viewer

\markboth{\spacedlowsmallcaps{Acronyms}}{\spacedlowsmallcaps{Acronyms}}

\chapter*{Acronyms}

\begin{acronym}[]
\acro{API}{Application Programming Interface: A specification intended to be used as an interface by software components to communicate with each other.}
\acro{CP}{Constraint Programming: Constraint programming is a programming paradigm wherein relations between variables are stated in the form of constraints.}
\acro{CSP}{Constraint Satisfaction Problem: Mathematical problems defined as a set of objects whose state must satisfy a number of constraints or limitations.}
\acro{COP}{Constraint Optimisation Problem: Can be defined as a regular constraint satisfaction problem in which constraints are weighted and the goal is to find a solution maximizing the weight of satisfied constraints.}
\acro{RSS}{Really Simple Syndication: A family of web feed formats used to publish frequently updated works—such as blog entries, news headlines, audio, and video—in a standardized format.}
% .*, Aa: [A-Z][A-Z]+
% .*: \\emph\{(\w)*
\end{acronym}  
                   
\endgroup

\cleardoublepage