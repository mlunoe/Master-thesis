%!TEX root = thesis.tex
\chapter{Conclusion} % (fold)
\label{ch:conclusion}
This paper solved the problem of personalising the editorial mix using Constraint Programming (CP). Different patterns in the editorial mix might emerge as new knowledge to the field is applied. However, the Constraint Personalisation Library makes it easy to extend the implemented solution with additional rules or changes to the existing ones. The fact that the library is implemented with a general purpose solver seems to make it possible to use it in many different contexts. The logic structure of how to define the problem should make it possible for other developers, that may not have knowledge of CP, to use the library in new contexts. The running time of the library is acceptable, but not optimal. However, it does make use of both probabilistic and logical approaches. The library is prepared for additional actions to optimise it and it is possible for the developer to define the problem to optimise the running time of the solver.

The implemented semantic analysis of articles does in practice seem to perform as it should, but a deeper analysis of this is needed in order to draw conclusions.
% section conclusion (end)