%!TEX root = thesis.tex
\chapter{Theory} % (fold)
\label{ch:theory}

\section{Personalisation}
\begin{quotation}
	``Personalization technology enables the dynamic insertion, customization or suggestion of content in any format that is relevant to the individual user, based on the user’s implicit behaviour and preferences, and explicitly given details.
	This can be dissected as:
	`Personalization technology enables the dynamic insertion, customization or suggestion of content' – personalization doesn’t just have to be product recommendations: it can also include inserting any content like images or text (e.g. displaying a golf-orientated banner for a returning golf supplies buyer), or customizing content that is already there (e.g. `Hi Joe, we've got some great movie suggestions for you!').
	`$\cdots$ in any format' – it isn’t restricted to the web. It can be implemented for any medium or touchpoint, such as emails, apps, instore kiosks, etc.
	`$\cdots$ that is relevant to the individual user, based on the user's implicit behaviour and preferences, and explicitly given details' – finally, the most important part. Personalization uses both implicit and explicit information, derived in two ways. Firstly, a visitor might explicitly declare some information, such as their gender or date of birth.''
\end{quotation}
\todonote{definition from elsewhere than wikipedia}



\section{Contraint Programming}
\subsection{Why Constraint Programming?}
\subsection{Constraint Programming In The Context Of Personalisation}
Hard constraints and soft constraints


Wiki: ``A constraint optimization problem can be defined as a regular constraint satisfaction problem in which constraints are weighted and the goal is to find a solution maximizing the weight of satisfied constraints.
Alternatively, a constraint optimization problem can be defined as a regular constraint satisfaction problem augmented with a number of `local' cost functions. The aim of constraint optimization is to find a solution to the problem whose cost, evaluated as the sum of the cost functions, is maximized or minimized.''
\todonote{Find reference that is not from wikipedia}





% section results (end)