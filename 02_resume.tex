%!TEX root = thesis.tex
\chapter{Resum\'e} %{Sammenfatning (Summary in Danish)}

%Projektet præsenterer den grundlæggende teori bag Constraint Programming (CP) og playlists. Strategien i arbejdet med constraints, logisk og funktionel programmering er anvendt til at opstille en Automatic Playlist Generator, som også benytter sig af en funktion til at sammenligne musiknumre. Produktet af projektet er et program i SML der genererer playlists ud fra forslag og forbud på sange fra et bibliotek. Funktionen til sammenligning er alene baseret på sammenligning imellem tempo og toner (key), som resulterer i forholdsvist ubruglige playlists. Programmet mangler sammenligninger imellem klangfarve (timbre), rytme og melodi, men den dynamiske tilgang gør at det let kan implementeres. Til sidst sluttes af den anvendte CP og local search tilgang viser sig effektiv til at løse problemet.