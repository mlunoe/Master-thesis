%!TEX root = thesis.tex
\chapter{Introduction} % (fold)
\label{ch:introduction}

%\todonote{? Første side er vigtig for karaktergivningen. ``Det er komplekst det her, Nå, der er et bud på det... osv.''}
\section{Problem description}
% Optimising the Editorial Mix for a Digital Newspaper using Constraint Programming
% Optimering af den Redaktionelle Sammensætning i en Digital Avis med brug af Constraint Programmering
% start 6/2
% aflevering 3/8
% General project objectives:
% The overall objective of the project is to make the student familiar with Constraint Programming, especially Constraint Optimised Problems, and its uses in the context of personalisation. A specific personalisation problem is implemented and solved using Constraint Programming and relevant metaheuristics. Finally, the project provides a conceptual basis for developing personal digital solutions and the uses for Constraint Programming.
% Learning objectives:
% Explore existing literature and summarise it.
% Analyse the applicability of personalisation in a given domain and derive features of the problem to be personalised.
% Find applicable metaheuristics for personalisation problems and refine them accordingly.
% Model the identified features as constraints in a Constrained Optimisation Problem. Solve the defined Constraint Optimisation Problem in Constraint Programming.
% Evaluate Constraint Programming as a tool for personalisation problems and compare the solution to existing solutions.
% Describe and discuss the project in a report and present it orally.
%
%
% Revised Problem Specification:
\subsection{Title}
Personalising the Editorial Mix for a Digital Newspaper using Constraint Programming

\subsection{Danish Title}
Personalisering af den Redaktionelle Sammensætning i en Digital Avis med brug af Constraint Programmering

\subsection{Time Frame}
Project start 6/2-2012, delivery 3/8-2012.

\subsection{General Project Objectives}
The overall objective of the project is to make the student familiar with personalisation and the concept of modelling user preferences in digital solutions. A specific personalisation problem is analysed and relevant design proposals are discussed. A chosen design is implemented and solved using Constraint Programming. The project provides a conceptual basis for the use of Constraint Programming in the context of developing personalised digital solutions. Finally, the project attempts to make personalisation more accessible for developers with the introduction of Constraint Programming to the field.

\subsection{Learning Objectives}
Explore existing literature and summarise it.

Analyse the applicability of personalisation in a given domain and derive features of the problem to be personalised.

Identify features in the domain that can be solved using Constraint Programming.

Model the identified features as constraints and solve them using Constraint Programming.

Refine the identified features into general guidelines for the use of Constraint Programming in the context of personalisation.

Evaluate Constraint Programming as a tool for personalisation problems and compare the solution to existing solutions.

Describe and discuss the project in a report and present it orally.
% end of Revised problem spec

%\footnote{In constraint satisfaction, constrained optimization seeks for a solution maximizing or minimizing a cost function, Wikipedia.}
%\footnote{Personalization involves using technology to accommodate the differences between individuals, Wikipedia.}
The overall idea is to determine the use of Constraint Programming (CP) as a tool to make the personalisation of digital solutions more accessible.
\todonote{Footnote: definition of CP and definition of personalisation}

The project will be divided into two main areas; i.e.~an assessment of the use of CP in the context of personalisation and a direct application of this in the form of a personal digital news paper, where CP is used to personalise the content and composition of a digital newspaper.

\subsection{Personalisation Challenges}
In an attempt to personalise the content of a newspaper, the report will try to analyse which preferences the users will have with respect to the content and composition of relevant articles. It will describe the search for articles to fit the user needs as an Constraint Optimisation Problem and try to solve it. What makes a newspaper is not only the accumulated content of its articles, but the editorial mix of them. ``Which articles should go where'' is just as important and the composition of newspaper should therefore go through an equal solving process.
\todonote{Footnote: definition of COPs}

\subsection{Algorithmic Challenges}
To be able to use and assess CP in the context of personalisation a full understanding must be acquired. Features that can be solved using CP will be modelled as a COP and solved. Furthermore, because the problem has a fixed budget for finding a solution, its algorithmic complexity will be analysed. The findings will be concluded in an evaluation of the applicability of personalisation problems in CP.

\subsection{Existing solutions}
Personalisation of digital solutions becomes more common everyday and users demand personalised solution to accommodate their needs. Many solutions to this problem already exists, but the role of CP within this domain has not been determined. This project seeks to explore CP as a tool to make the personalisation of digital solutions more accessible.
%
%\subsection{Subproblem specification}
%Before evaluating the use of COPs with respect to personalisation it needs to be determined whether the problem can actually be described as a COP. Moreover, the editorial constraints needs to be defined. These will constitute a knowledge base of ground rules from which a paper will be generated. The article will look into current media,~i.e.\ radio programmes, television programmes, magazines and of cause news papers, to try to specify rules of layout and succession of articles based on metadata. The first part of this article will address this problem.
%
%A crucial part of the project is also to lay down the means of bringing content to the paper. A normal way to do this would be to use feeds or to scrape web sites (extracting information from web sites) of articles. The pros and cons of each method and further options needs to be addressed and an optimal solution chosen. This will constitute the second part of the article.
%
%
%
%
%bedste måde at inddrive den information vi skal bruge for at kunne arbejde med det og generere den digitale avis der passer dig bedst
%
%analyse - cop i dybden medier kan - synagien i et layout genskabes fra links og tweets?, design, implementation - web, anvende, test (evaluere). Realtion til playlist - som går på at få en stemning kommunikeret. Bygge en oplevelse - hver artikel bidrager til helheden. tværsnit. analyse=model+brugerpræferencer. topic (flere) vs. genre (én). grænseværdi for topics (hvor mange giver mening) granulalitet. genre i første omgang. analysere sig frem til sammensætning af ord => definere topics - distributioner af ord. mappe ord fra topic knowledge base til artikel. opbygge og glemme viden gradvist (1 måned eller 1 år?). Udvælge punkter der kunne være at undersøge.
%
%\subsection{General Project Objectives}
%The overall objective of the project is to make the student familiar with Constraint Programming, especially Constraint Optimised Problems, and its uses in the context of personalisation. A specific personalisation problem is implemented and solved using Constraint Programming and relevant metaheuristics. Finally, the project provides a conceptual basis for developing personal digital solutions and the uses for Constraint Programming.
%
%\subsection{Learning Objectives}
%Explore existing literature and summarise it.
%
%Analyse the applicability of personalisation in a given domain and derive features of the problem to be personalised.
%
%Find applicable metaheuristics for personalisation problems and refine them accordingly.
%
%Model the identified features as constraints in a Constrained Optimisation Problem.
%
%Solve the defined Constraint Optimisation Problem in Constraint Programming.
%
%Evaluate Constraint Programming as a tool for personalisation problems and compare the solution to existing solutions.
%
%Describe and discuss the project in a report and present it orally.

\section{Motivation}
The development of the Internet from a distributer of information to a library of digital applications has deeply integrated the users in every step of an applications lifetime. It has even become harder to distinguish between super users and developers, applications are branched and modified according to every need and authors can therefore no longer predict which use his or her application can be to another user - nor should he/she have to.

User preferences are very diverse and it is therefore hard to accommodate every individual in a single solution. A digital solution must be bound to a specific domain, but must also be open for novel use.

Constraint Programming offers a more natural way of defining problems,~i.e. what should be solved (and not how). This makes the modelling of problems very intuitive, once setup, and can afterwards easily be extended and modified.
\todonote{reference to what - not how}

\todonote{Something about introducing an automated editorial mix to the digital newspaper. Maybe about bringing rss-readers and digital newspapers closer together.}


%knowledge
%comprehension
%application
%analyse
%synthesise
%evaluate
% section introduction (end)